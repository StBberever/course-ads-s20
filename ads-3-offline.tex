\section{Об оффлайн деревьях поиска: введение}

[Я пишу здесь свою чепуху, а потом пытаюсь понять, как она вклеивается в уже написанное]

\subsection{Оффлайн деревья поиска и графическое представление}

Пусть у нас есть дерево над ключами $1$, $2$, \ldots, $n$ и последовательность $S = (s_1, s_2, \ldots, s_m)$ запросов поиска. Запросы нам известны заранее и мы хотим построить такое дерево, чтобы минимизировать суммарное время, потраченное на то, чтобы ответить на эти запросы. Мы будем считать, что $m \geqslant n$, более того, мы потрогали каждый ключ хотя бы один раз. [Не знаю, нужно ли это на самом деле.]
При этом мы разрешаем перестраивать дерево с помощью вращений в процессе 
ответа на запросы.

Нам разрешено делать следующие вещи:
\begin{itemize}
\item[1.] Переходить по указателю.
\item[2.] Делать одинарное вращение с центром в этой вершине (zig, он же zag).
\end{itemize}

Начинаем при этом мы всегда в корне, а для каждого запроса хотим посетить вершину
с соответствующим ключом.

\newdefn{Последовательность таких действий для фиксированной последовательности 
запросов $S$ называется \emph{BST-алгоритмом}.}

\newdefn{Цена операции поиска~--- количество посещённых узлов. Поскольку для того, чтобы сделать вращение в вершине, нам нужно её посетить, учитывать количество вращений не нужно, если мы считаем с точностью до константы.}

\newdefn{$\opt(S)$~--- минимальная суммарная цена выполнения $S$, если мы заранее знаем $S$ и можем в связи с этим выбирать, какие именно операции вращения
мы будем делать, а какие~--- нет.} 

Значение 
$\opt(S)$ мы не умеем искать (даже с точностью до мультипликативной
константы) за полином. Впрочем, опровергать возможность вычисления $\opt(S)$ 
за полином мы тоже не умеем, даже в предположении $\mathrm{P} = \mathrm{NP}$. Ясно,
что задача о проверке неравенства $\opt(S) \leqslant k$ лежит в $\mathrm{NP}$.

Это можно переформулировать в геометрических терминах. Рассмотрим координатную
плоскость с ключами по оси $x$ и моментами времени (то есть номерами запросов) по 
оси $y$. Для данной последовательность запросов $S$ отметим все точки $(s_i, i)$ 
жирной точкой на плоскости (мы обязаны посетить ключ $s_i$ при обработке $i$-того
запроса, так как мы должны его найти). Также отметим крестиком все точки $(k, i)$ такие,
что мы посетили ключ $k$ при обработке $i$-того запроса. Понятно, что стоимость данной
последовательность операций~--- количество точек, которые мы отметили жирной точкой
или крестиком. 

\newdefn{Множество отмеченных точек~--- \emph{графическое представление} данного BST-алгоритма. У разных BST-алгоритмов могут быть одинаковые представления.}	

На изображении слева видна картинка, которая получится, если для дерева на изображении
справа применить последовательность запросов $S = (2, 2, 4, 5, 3)$ и при этом не совершать
никаких вращений. От крестика, который обведён в кружочек, можно избавиться, если 
при обработке четвёртого запроса сделать операцию \textrm{rotate 3} и получить нарисованное ниже дерево (в таком дереве для обработки запроса \textrm{найти ключ 3} не нужно посещать вершину с ключом $2$, так как вершина с ключом $3$ уже является корнем).

Про splay-деревья верят, что они оптимальны с точностью до мультипликативной константы,
то есть что они посещают $O(\opt(S))$ вершин при обработке любого списка запросов $S$. Это достаточно круто, так как splay-деревья не знают будущего,
в отличие от оптимальное алгоритма. Однако, доказывать это про splay-деревья не умеют,
но есть другие деревья, про которые это умеют доказывать.

\newdefn{Множество целых точек на плоскости $E$ называется \arbs,
если для любых точек $a$ и $b$ из $E$ верно хотя бы одно из следующих трёх свойств: $x(a) = x(b)$, $y(a) = y(b)$ или прямоугольник, натянутый на точки $a$ и $b$, как на противоположные углы, содержит точку из $E$.}

\subsection{Эквивалентность BST-алгоритмов и \arbs множеств}

\begin{theorem} Если множество точек $E$ может быть отмечено каким-то BST-алгоритмом,
то оно \arbs.
\end{theorem}
\begin{proof} Предположим противное. Пусть мы нашли две точки $a$ и $b$, при этом
$x(a) \neq x(b)$, $y(a) \neq y(b)$ и внутри $\mathrm{rect}(a, b)$ нет других точек $E$. 
Не умаляя общности, $i =: y(a) < y(b) =: j$. Пусть $c$~--- наименьший общий предок $a$ и $b$ в момент времени $i$.

Есть два случая:
\begin{itemize}
\item[1.] Если $c = a$, то мы должны были посетить $a$ в какой-то момент из отрезка $(i + 1, j]$. Действительно, раз $a$ является предком $b$ в момент времени $i$, то либо $a$~--- всё ещё предок $b$ \emph{перед} моментом времени $j$ (и тогда мы должны посетить $a$ просто для того, чтобы дойти до $b$), либо вершина $a$ перестала быть предком $b$ в какой-то из моментов на отрезке $(i + 1, j)$, а для этого мы должны были посетить её и сделать вращение в её ребёнке.

\item[2.] Если $c \neq a$, то $a$ и $b$ лежат в разных поддеревьях $c$. Следовательно, по свойству двоичного дерева поиска, $x(c) \in \langle x(a), x(b) \rangle$ (ключ вершины $c$ должен лежать между ключами вершин $a$ и $b$; здесь $\langle s, t \rangle$ это либо $[s, t]$, если $s < t$, либо $[t, s]$ в противном случае). Раз мы посетили $a$ при обработке $i$-того
запроса, то мы посетили и её предка $c$, следовательно множество $E$ содержит точку $(x(c), y(c)) = (x(c), i)$, а она лежит в искомом прямоугольнике.
\end{itemize}
\end{proof}

Стоит заметить, что мы доказали более сильный факт: если $x(a) \neq x(b)$ и $y(a) \neq y(b)$, то есть точка из $E \setminus \{a \}$, которая попала на одну из сторону $\rect(a, b)$, смежную с $a$ (то, какая это сторона, зависит от того, $c = a$ или $c \neq a$). Аналогичное утверждение верно для $E \setminus \{ b \}$ и $b$. 

Дальше мы будем постоянно пользоваться следующей леммой, утверждающей, что описанное в прошлом абзаце условие верно для любого \arbs множества, а не только для тех, которые
являются графическим представлением BST-алгоритма (позже мы поймём, что каждое
\arbs множество~--- графическое представление какого-то BST-алгоритма, но не будем торопить события).

\begin{lemma} Если $E$~--- \arbs, то для любых $a$ и $b$ из $E$, таких что $x(a) \neq x(b)$ и $y(a) \neq y(b)$, существует точка из $E \setminus \{a \}$, которая попадает на одну из сторон 
$\rect(a, b)$ 
\end{lemma}
\begin{proof} В одну сторону понятно, так как strongly \arbs сильнее (на точку внутри прямоугольника накладывается больше условий). 

В другую сторону: пусть у нас есть $\rect(a, b)$ для точек $a$ и $b$, удовлетворяющих 
условиям $x(a) \neq x(b)$ и $y(a) \neq y(b)$. Так как $E$~--- \arbs множество, то есть
$c \in \rect(a, b)$, $c \neq a$ и $c \neq b$. Есть два случая:
\begin{itemize}
\item[1.] $x(c) = x(a)$ или $y(c) = y(a)$. Тогда $c$ лежит на одной стороне $\rect(a, b)$
с точкой $a$. То есть для прямоугольника $\rect(a, b)$ и точки $a$ выполняется сильное
\arbs свойство.
\item[2.] $x(c) \neq x(a)$ и $y(c) \neq y(a)$. Тогда в $\rect(a, c)$ тоже есть точка из $E \setminus \{a, c \}$ по обычному \arbs свойству, при этом $\rect(a, c)$ строго меньше
$\rect(a, b)$. Будем повторять процесс (возьмём точку $d \neq a$, $d \neq c$ из $\rect(a, c)$,
и так далее), пока неизбежно не выполнится случай $1$.
\end{itemize}
\end{proof}

Немного удивительно, но верно и обратное следствие.

\newdefn{\emph{Декартово дерево} (\textrm{treap}) на парах $(\mathrm{key}_i, \mathrm{priority}_i)$~--- это
сбалансированное двоичное дерево поиска по ключам (первым элементам пар) и куча на минимум по приоритетам (вторым элементам пар). Если все ключи и приоритеты различны, то для есть всего одно такое, иначе их может быть несколько.}

\begin{theorem} Если $E$~--- \arbs, то существует BST-алгоритм, графическое представление
которого в точности равно $E$. Формально говоря, нужно ещё не забыть наложить
условие, что множество $y$-координат точек из $E$~--- в точности отрезок целых чисел $[1, n]$ для какого-то $n$. Это соответствует тому, что при каждом запросе мы должны обязательно посетить корень, то есть хотя бы одну вершину.
\end{theorem}
\begin{proof} В момент времени $i$ наше дерево будет \emph{каким-то} (не любым, а именно каким-то; то есть ``существует последовательность'', а не ``для любой последовательности'') декартовым деревом на парах $(x, N(x, i))$, где $N(x, i)$~--- минимальное такое $j \geqslant i$, что $(x, j) \in E$ или $+\infty$, если таких $j$ нет. Интуитивно, $N(x, i)$ должно быть первым моментом времени, начиная с $i$, когда мы посетим ключ $x$. $T_1$~--- какое-то декартово дерево, хотим перестроить $T_i$ в $T_{i+1}$, посетив только вершины, которые нам разрешено посещать в момент времени $i$
(то есть вершины с такими ключами $x$, что $(x, i) \in E$; назовём множество всех таких вершин $\tau_i$).

Вершины, которые мы можем посещать в момент времени $i$~--- какой-то связный кусок $T_i$, содержащий корень $T_i$. Почему? Потому что у всех вершин $T_i$ приоритет равен $N(x, i)$, то есть хотя бы $i$, а у вершин из $\tau_i$ приоритет равен ровно $i$. При этом только
у вершин из $\tau_i$ приоритет поменяется на что-то новое (так как для других ключей $N(x, i) = N(x, i + 1)$). Нам нужно как-то поменять приоритеты вершин из $\tau_i$ и перестроить дерево, вращая только вершины из $\tau_i$. 

Любое двоичное дерево поиска можно переделать в любое двоичное дерево поиска на тех же ключах с помощью линейного количества вращений. Это проще всего понять, если воспользоваться биекцией между триангуляциями выпуклого $n$-угольника и двоичными деревьями: на языке триангуляций вращение означает операцию \textrm{flip} (поменять в четырёхугольнике с проведённой диагональю проведённую диагональ) и достаточно понятно, как привести любую триангуляцию за линейное число \textrm{flip}-ов к триангуляции, в которой
все треугольники исходят из одного угла (соответствует бамбуку). 

Поэтому нас только волнует, что после того, как мы перестроим часть $T_i$ с вершинами из $\tau_i$ в состояние, в котором она должна находиться в $T_{i+1}$, не появится вершин, нарушающих свойство кучи. Пусть в построенном нами $T_{i+1}$ есть вершина с парой ``ключ-приоритет'' $(y, j)$ не из $\tau_i$ и у неё есть предок из $\tau_i$ с парой $(x, k)$. Раз эти вершины нарушают свойство кучи, то $j < k$ (полностью внутри перестроенной области и полностью вне неё ничего сломаться не могло: первую мы перестраивали, сохраняя свойство кучи, а вторую не трогали).

Не умаляя общности, $x < y$. Посмотрим на точки $(x, i)$ и $(y, j)$ из $E$ и натянутый на них прямоугольник $\rect((x, i), (y, j))$. На вертикальной стороне от $(x, i)$ до $(x, j)$ нет ничего из $E \setminus \{(x, i) \}$ по определению, так как $N(x, i + 1) = k > j$. Следовательно, по усиленной версии \arbs-свойства, есть точка $(c, i)$ на
стороне от $(x, i)$ до $(y, i)$. Это значит, что $c \in \tau_i$. Все наши операции при перестройке $T_i$ в $T_{i+1}$ были вращениями: они могли сломать свойство кучи, но не свойство двоичного дерева поиска. Поэтому, ключ $c$ всё ещё лежит между ключами $x$ и $y$. Но вершина с ключом $y$~-- отец вершины с ключом $x$ в $T_{i+1}$, следовательно вершина с ключом $c$ находится где-то в поддереве $y$ в дереве $T_{i + 1}$ (см. картинку). Это невозможно, так как $c \in \tau_i$ и должна была остаться в связном куске $T_{i + 1}$, содержащем корень (но не осталась, так как она отделена вершиной $y \notin \tau_i$ от вершины $x \in \tau_i$). Противоречие.
\end{proof}

\subsection{``Онлайн-эквивалентность'' BST-алгоритмов и \arbs множеств}

Только что мы получили оффлайн-алгоритм, который, зная \arbs множество $E$, строит BST-алгоритм с графическим представлением $E$. Утверждается, что есть \emph{онлайн}-алгоритм который, получая не всё $E$ сразу, а по строкам (получил $\tau_1$, сделал нужные операции, получил $\tau_2$, сделал нужные операции, и так далее)), строит BST-алгоритм со стоимостью $O(|E| + n)$ 	
(получить в точности графическое представление $E$ не получится, ухудшения на мультипликативную константу не избежать).

Нам понадобится немного необычная структура данных.

\newdefn{
\emph{split-дерево} (\textrm{split-tree})~--- это абстрактная структура данных, состоящая из \emph{внутреннего двоичного дерева поиска} на имеющихся ключах и какой-то \emph{дополнительной информации}, которая может иметь любую природу. При этом она должна уметь поддерживать две операции:
\begin{itemize}
\item[1.] \textrm{make\_tree}($x_1$, $x_2$, \ldots, $x_n$)~--- по отсортированному массиву ключей построить структуру данных, при этом внутреннее дерево поиска должно быть двоичным деревом поиска на данных ключах;
\item[2.] \textrm{split\_tree}($x$)~--- найти ключ $x$ в двоичном дереве поиска (гарантируется, что он там есть), с помощью вращений поднять его в корень, удалить его и вернуть два новых split-дерева: левое и правое поддеревья корня (в левом все ключи меньше $x$, а в правом все ключи больше $x$).
\end{itemize}
При этом разрешается тратить суммарное только $O(n)$ времени на построение (\textrm{make\_tree})
и полное разрушение ($n$ операций \textrm{split\_tree}) дерева.
}

\begin{remark} Небольшое отступление о природе split-дерева. Операции ``постройте структуру по списку чисел'' и ``найдите данное число в структуре, удалите его и разбейтесь на <<до>> и <<после>>'' можно легко реализовать с помощью односвязного списка и хэш-таблицы или кучи других подобных методов.

Но суть split-дерева не в этом. Суть split-дерева в том, что оно реализует операцию \textrm{split\_tree} \emph{физически} на внутреннем двоичном дереве поиска с помощью вращений в точности так, как описано. Вся дополнительная информация, которую мы храним, существует не для того, чтобы отвечать на какие-то запросы об элементах структуры, а только для того, чтобы лучше понимать, как и когда совершать дополнительные вращения, кроме тех, которые нам нужны, чтобы пригнать ключ $x$ в корень.

Нас интересует не столько время, которые мы потратили, сколько число вершин во внутреннем двоичном дереве, которые мы затронули. Если бы мы могли потратить $O(n^2)$ времени, но затронуть вершины только $O(n)$ раз в процессе полного разрушения внутреннего дерева, это бы нас более-менее устроило. Но оказывается, что мы можем потратить $O(n)$ времени (и, следовательно, лишь $O(n)$ раз затронуть вершины внутреннего дерева). Раз можем, то почему бы и не воспользоваться чуть лучшей версией алгоритма?

Я не буду воспроизводить принцип работы split-дерева, так как он не очень важен. Узнать его можно в исходной статье~\cite{DHIKP}. Более того, есть гипотеза (см. статью Лукас~\cite{Luc88}), что в качестве split-дерева можно использовать обыкновенное splay-дерево без дополнительной информации (и, соответственно, не делать никаких вращений, кроме тех, которые нужны, чтобы пригнать ключ в корень), но доказывать это не умеют.
\end{remark}

Теперь мы будем на каждом шаге строить не обычное декартово дерево, а обобщённое декартово дерево (определение в следующем абзаце) $G_i$
есть обобщённое декартово дерево (с отличием, что теперь мы просим, чтобы декартово дерево было кучей на \emph{максимум} по приоритетам), построенное на парах $(x, \rho(x, i))$, где $\rho(x, i)$~--- максимальное такое $j < i$, что $(x, j) \in E$ или $-\infty$, если таких нет. То есть $\rho(x, i)$~--- последний момент строго перед $i$, когда ключ $x$ был задет. Фактически, мы повернули вспять течение времени и сделали так, что теперь вершины с большими $\rho(x, i)$ находятся выше в дереве (раньше~--- вершины с меньшими $N(x, i)$). Однако, не всё так просто, так как теперь вершины, у которых мы меняем приоритет расположены внутри дерева, на первый взгляд, как-то случайно.

\newdefn{\emph{Обобщённое декартово дерево} (\textrm{general treap})~--- это на самом деле обычное декартово дерево, на которое наложено несколько дополнительных ограничений:
\begin{itemize}
\item[1.] Вершины с одинаковым приоритетом объединяются в \emph{суперузлы}. Каждый суперузел~--- связное подмножество дерева. Для одного приоритета может быть несколько суперузлов, но они не связаны между собой (то есть соседние вершины с одинаковым приоритетом обязаны попасть в один и тот же суперузел).
\item[2.] Каждый суперузел~--- split-tree (точнее, внутреннее двоичное дерево для split-tree).
Гарантируется, что суперузлы создаются с помощью операции \textrm{make\_tree}, а потом
постепенно разрушаются с помощью операций \textrm{split\_tree}. Так как внутри суперузла все приоритеты одинаковые, то любое двоичное дерево поиска будет удовлетворять условию кучи. 
\end{itemize}
}

Изначально, $G_1$ состоит из одного суперузла с приоритетом $-\infty$, который мы строим с помощью \textrm{make\_tree}. 
Как получить $G_{i+1}$ из $G_i$? Для этого нужно взять все
вершины из $\tau_i$, так как только для них $\rho(x, i+1) \neq \rho(x,i)$ (а именно, $\rho(x, i + 1) = i$ для $x \in \tau_i$; для других вершин $\rho(x, i + 1) = \rho(x, i) < i$), 
вырезать их из своих суперузлов с помощью \textrm{split\_tree} и создать новый суперузел с приоритетом $i$ с помощью \textrm{make\_tree}. Строгое понимание этих слов (в частности, то, как мы поддерживаем при всех этих операциях свойства обобщённого декартово дерева и даже то, почему $G_{i+1}$ вообще окажется хотя бы обычным декартовым деревом) отложим на потом, а пока поймём, что мы не можем посетить какие-то абсолютно левые вершины.

[Оставь надежду, всяк сюда входящий.]

Пусть мы посетили (то есть $x \in \tau_i$) вершину с парой ``ключ--приоритет'' $(x, k)$ в $G_i$. Пусть отец её суперузла (в $G_i$, всё пока в $G_i$)~--- суперузел $P$ с приоритетом $j > k$. 
Пусть $\mathrm{succ}(P, x)$~--- наименьший ключ в $P$, больший $x$, $\mathrm{pred}(P, x)$~--- наибольший ключ в $P$, меньший $x$.
Утверждается, что мы их посетили (если они существуют), то есть $\mathrm{succ}(P, x) = +\infty$ или $\mathrm{succ}(P, x) \in \tau_i$, и, аналогично, $\mathrm{pred}(P, x) \in \tau_i \cup \{-\infty\}$.

[Тут нужна картинка, это не очень сложно, но без картинки у меня взрывается мозг.]
Действительно, пусть $\mathrm{succ}(P, x) < +\infty$ и $\mathrm{succ}(P, x) \notin \tau_i$, то есть $(\mathrm{succ}(P, x), j + 1), (\mathrm{succ}(P, x), j + 2), \ldots, (\mathrm{succ}(P, x), i) \notin E$. Тогда, так как $(x, i) \in E$ и $(\mathrm{succ}(P, x), j) \in E$, то в $E \setminus \{\mathrm{succ}(P, x)\}$ есть точка на стороне $(\mathrm{succ}(P, x), j)--(x,j)$ прямоугольника $\rect((x, i), \mathrm{succ}(P, x))$. Следовательно, есть такое $y \in [x, \mathrm{succ}(P, x))$, что $(y, j) \in E$. Так как $\rho(x, i) = k < j$, то $(x, j) \notin E$, откуда $y$ лежит строго между
$x$ и $\mathrm{succ}(P, x)$. Получается, что $\rho(y, i) \geqslant j$. Почему это странно? Получается, что ключ $y$ каким-то образом лежит между ключом $\mathrm{succ}(P, x)$ и ключом $x$. Это означает, что он лежит в поддереве наименьшего общего предка 
$\mathrm{succ}(P, x)$ и $y$. Этот наименьший общий предок~--- какая-то вершина из $P$ (потому что $\mathrm{succ}(P, x)$ лежит в $P$ по определению, а $P$~--- отец суперузла, в котором лежит $x$), следовательно его приоритет \emph{не больше} приоритета $P$, то есть $j$. С другой стороны, мы уже доказали, что $\rho(y, i) \geqslant j$. Отсюда, приоритет $y$ в точности равен $j$ и $y$ лежит в $P$. Но это противоречит определению $\mathrm{succ}(P, x)$, так как $x < y < \mathrm{succ}(P, x)$. Аналогично с $\mathrm{pred}(P, x)$.

Что мы получили? Как минимум то, что множество посещённых \emph{суперузлов} (то есть таких суперузлов, в которых есть вершина из $\tau_i$), образуют связное множество, содержащее корень. Аналогичное утверждение про \emph{вершины} неверно, но оно нам и не понадобится. Сейчас нам понадобится понять ещё одну интересную особенность split-дерева.

\begin{remark} Split-дерево в процессе своей работы производит какие-то операции вращения внутреннего двоичного дерева поиска. Но эти операции вращения (выполнить вращения в какой-то вершине) можно совершать и со связными подмножествами большего дерева поиска, в нашем случае обобщённого декартова дерева (которое, как я напомню, является обычным декартовым деревом с какой-то дополнительной структурой).

То есть внутренние деревья поиска наших split-деревьев~--- какие-то куски нашего
обобщённого декартового дерева, а именно, суперузлы. В частности, эти внутренние
деревья не нужно хранить отдельно, так как они сохранены в нашем декартовом дереве.
Когда split-дерево говорит, что нам нужно сделать вращение во внутреннем дереве, мы делаем вращение в соответствующей вершине нашего большого дерева (в процессе вращений
оно могло временно перестать быть \emph{декартовым} деревом, но всё ещё остаётся \emph{деревом поиска}). 

При таком понимании у удаления вершины (операции \textrm{split\_tree}($x$)) появляется
следующая интерпретация: мы не столько \emph{удаляем} вершину с ключом $x$, сколько пригоняем её в корень куска большого дерева, соответствующего нашему split-дерева, прекращаем ассоциировать её с нашим split-деревом и чисто формально (дополнительную информацию, если она есть, нужно будет пересчитать, но менять в структуре большого дерева ничего не надо) разбиваем наше split-дерево на два. 

После такой операции наша вершина перестаёт быть ассоциированой с \emph{каким-либо} split-деревом, поэтому мы больше не будем делать вращений с центром в ней до того, как перейдём к $(i+1)$-ой итерации процесса (построению $G_{i+2}$ по $G_{i+1})$. 

Однако,
это не помешает ей дойти до корня (здесь-то мы и воспользуемся доказанным ранее
условием про $\mathrm{succ}(P, x)$ и $\mathrm{pred}(P, x)$). 
\end{remark}

Наша цель состоит в том, чтобы небольшим количеством вращений пригнать все
ключи из $\tau_i$ в какое-то связное множество вершин, содержащее корень, 
не сломав при это условие кучи на других вершинах.
После этого мы вызываем \textrm{make\_tree} от этих вершин и делаем их приоритеты
равными $i$. Внутреннее двоичное дерево, которое нам вернёт \textrm{make\_tree}
может отличаться от структуры двоичного дерева поиска на этих вершинах, которая получилась после того, как мы пригнали их всех наверх, но, как мы знаем, мы можем переделать одно в другое за линейное количество вращений.

Как мы пригоняем все вершины наверх? На удивление просто: пройдёмся по суперузлам в порядке от более глубоких к менее глубоким. Внутри каждого суперузла пройдёмся (скажем, в порядке возрастания, но это должно быть неважно) по всем ключам $x$ из этого суперузла, попавшим в $\tau_i$ и для каждого из них сделаем операцию 
\textrm{split\_tree}($x$). 

Почему это работает? Внутри каждого суперузла первый рассмотренный ключ приедет в корень суперузла, вторая~--- в один из корней двух внутренних деревьев, полученных из исходного, то есть в одного из детей корня суперузла, и так далее. То есть все рассмотренные ключи в итоге приедут в какое-то связное множество, содержащее корень суперузла. Таким образом, каждый ключ ``доезжает'' до корня своего суперузла ``своим ходом''.

Однако, как мы уже отметили ранее, мы перестаём делать вращения в вершине после того, как она приехала наверх своего суперузла. Раз сама она дальше проехать не может, то её должны
дальше довезти друзья (звучит позитивно)! 

Чтобы понять главную идею, рассмотрим случай, когда мы затрагиваем всего два суперузла: суперузел-корень (назовём его $P$) и одного из его суперузлов-сыновей. При этом в сыне мы затронули только один ключ $x$. Сперва мы пригоняем ключ $x$ в корень суперузла сына.
После этого ключи из $P$, вместе с ключом $x$ образуют правильное двоичное дерево поиска.
Как мы доказали раньше, $\mathrm{succ}(P, x)$ и $\mathrm{pred}(P, x)$ лежат в $\tau_i$. Когда мы пригоняем вершины из $\tau \cap P$ мы на самом деле разбиваем все оставшиеся вершины из $P$ на поддеревья в зависимости от того, как они сравниваются с вершинами $\tau \cap P$. Но теперь в одном из этих поддеревьев появляется гостья, которой раньше не было: вершина с ключом $x$. Это поддерево раньше было пустым, так как соответствовало вершинам из $P$ с ключами из интервала $(\mathrm{pred}(P, x), \mathrm{succ}(P, x))$, а таких нет по определению $\mathrm{pred}$ и $\mathrm{succ}$. А теперь в этом поддереве будет одна вершина с ключом $x$. Значит, её отец лежит в верхнем связном куске, состоящем из вершин с ключами из $P \cap \tau_i$. Значит, мы можем подклеить вершину с ключом $x$ к этому куску с сохранением связности. 

В общем случае, происходит следующее: внутри каждого суперузла затронутые (то есть из $\tau_i$) вершины этого суперузла собираются в одну большую группу наверху ``своим ходом''. Более того, все группы, пришедшие из суперузлов-детей тоже подклеятся к этой большой группе. В итоге все эти группы постепенно едут наверх и постепенно склеиваются, в итоге склеиваясь в один большой снежный ком в самом верху большого дерева. Мы это, собственно и хотели доказать. 

Всего мы сделали $O(|E| + n)$ вращений. Действительно, на $i$-том шаге мы делаем $|\tau_i|$ операций \textrm{split\_tree} (амортизированно $O(1)$ времени) и одну операцию \textrm{make\_tree} на $|\tau_i|$ вершинах ($O(|\tau_i|$ времени). Дополнительное слагаемое $O(n)$ появляется из-за амортизации: каждое ещё не разрушенное split-дерево могло ``съесть'' $O(\texttt{своего размера})$ операций (проще всего это понять, если $|E|$ близко к нулю).
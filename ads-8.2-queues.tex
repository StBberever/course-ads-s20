\section{Ретроактивность для конкретных структур данных}
\subsection{Двунаправленная очередь}

Двунаправленная очередь — это структура данных, хранящая некоторые элементы и поддерживающая следующие операции:

\begin{center} \begin{tabular}{ll}
	{\tt update} & $\text{pushL} (x)$: вставить элемент слева \\
		& popL: удалить элемент слева \\
		& $\text{pushR} (x)$: вставить элемент справа \\
		& popR: удалить элемент справа \\
	{\tt query} & Узнать крайний правый, крайний левый элементы
\end{tabular} \end{center}

\begin{theorem} [\cite{demaine2007retroactive}, Теорема 7]
	Двунаправленную очередь можно сделать полно ретроактивной, так что ретро-обновления и ретро-запросы будут занимать время $O \ll \log m \rr$, а запросы о текущем состоянии структуры — $O \ll 1 \rr$.
\end{theorem}

\begin{proof} \item \vspace{-7mm}
\paragraph{Описание структуры данных} Элементы текущей версии очереди будут храниться в массиве $A$. Также нам потребуются числа $L$ и $R$ — индексы левого и правого краёв рабочего участка массива $A$. Если очередь за последние несколько операций становилась короче, то за этими левыми и правыми краями будет оставаться что-то написанное, мы не будем его стирать, но и не будем мыслить его частью очереди.

	В первом приближении понятно, как делать push и pop в настоящий момент времени. Для push слева уменьшим на единицу индекс левого края и запишем на левый край новый элемент (возможно, поверх чего-то старого); для pop слева просто увеличим на единицу индекс края. Справа — симметрично.

	Рассмотрим два списка: $U_L$ и $U_R$. Это отсортированные по времени списки операций, соответственно, с правым и левым краями очереди. У каждой операции в этих списках есть вес: push имеет вес $+1$, pop — $-1$. Заметим, что индекс, например, правого края равен сумме весов всех операций в списке $U_R$.

	Сделаем из $U_L$, $U_R$ сбалансированные деревья поиска: операции будут храниться в листьях, а каждый узел будет хранить сумму весов всех операций его поддерева. Значение $R$ в момент, когда была сделана какая-то операция, можно посчитать, сложив значения, хранимые левыми братьями узлов, в которые мы спускались направо по пути к означенной операции. \vspace{-4mm}

\paragraph{Реализация ретро-операций} При ретро-операции в момент $t$ спустимся к этому моменту времени в соответствующем дереве ($U_L$ или $U_R$) и запишем операцию в новом листе. При этом прибавим вес операции к суммам во всех узлах, через которые мы спускались.

	Чтобы реализовать ретро-запрос, в каждом узле деревьев поиска будем также хранить наибольшее и наименьшее значения префиксных сумм, достигаемых в поддереве этого узла. Эти значения говорят нам, в каких пределах были левый и правый край рабочего участка очереди на соответствующем отрезке времени.

	Дан момент времени $t$, нужно выяснить, где в этот момент находился конец рабочего участка массива и что там было записано. Индекс $i$ конца рабочего участка мы можем посчитать, спустившись по дереву, как описано выше.

	Пусть мы работаем с правым краем очереди. Найдём в $U_R$ последнюю операцию перед моментом времени $t$. Поднимемся из её листа в корень дерева, а затем спустимся обратно, каждый раз выбирая правое поддерево из тех, для которых $i$ находится между наибольшим и наименьшим значениями префиксных сумм, хранящимися в вершине поддерева.

	Такой выбор поддерева гарантирует нам то, что мы попадём в некоторый элемент, хранящийся в нужной нам ячейке массива, при этом данный элемент появился там на нужном нам {\it переписывании} — то есть, он написан поверх элемента, который был на этом месте раньше, чем время $t$, и его ещё не успел сменить элемент, который появится там позже. Действительно, условие «находиться между максимумом и минимумом префиксных сумм» равносильно условию «оставаться в рабочем участке массива, не будучи перезаписанным другим элементом».
\end{proof}

\subsection{Очередь с приоритетом}
\definecolor{oldq}{RGB}{245,90,30}
\definecolor{newq}{RGB}{50,90,180}
\definecolor{mostq}{RGB}{70,220,170}

\begin{figure} \centering
\tikz[scale=0.29]{
    \foreach \t in {0,...,49} {\draw[gray,opacity=0.32] (\t,0) -- (\t,21.7);}
    \foreach \t in {0,...,31} {\draw[gray,opacity=0.32] (0,0.7 * \t cm) -- (49,0.7 * \t cm);}
    \draw[very thick,mostq] (33,0) -- (33,21.7);
        \draw[thick] (0,0.7 * 31) -- (1,0.7 * 31);        \draw[oldq,very thick] (1,0.7 * 31) -- (1,0);
        \draw[thick] (3,0.7 * 11) -- (5,0.7 * 11);        \draw[oldq,very thick] (5,0.7 * 11) -- (5,0);
        \draw[thick] (4,0.7 * 14) -- (6,0.7 * 14);        \draw[oldq,very thick] (6,0.7 * 14) -- (6,0);
        \draw[thick] (2,0.7 * 18) -- (7,0.7 * 18);        \draw[oldq,very thick] (7,0.7 * 18) -- (7,0);
        \draw[thick] (8,0.7 * 1) -- (11,0.7 * 1);        \draw[oldq,very thick] (11,0.7 * 1) -- (11,0);
        \draw[thick] (10,0.7 * 3) -- (12,0.7 * 3);        \draw[oldq,very thick] (12,0.7 * 3) -- (12,0);
        \draw[thick] (13,0.7 * 6) -- (16,0.7 * 6);        \draw[oldq,very thick] (16,0.7 * 6) -- (16,0);
        \draw[thick] (21,0.7 * 2) -- (24,0.7 * 2);        \draw[oldq,very thick] (24,0.7 * 2) -- (24,0);
        \draw[thick] (18,0.7 * 7) -- (27,0.7 * 7);        \draw[oldq,very thick] (27,0.7 * 7) -- (27,0);
        \draw[thick] (9,0.7 * 9) -- (28,0.7 * 9);        \draw[oldq,very thick] (28,0.7 * 9) -- (28,0);
        \draw[thick] (26,0.7 * 10) -- (29,0.7 * 10);        \draw[oldq,very thick] (29,0.7 * 10) -- (29,0);
        \draw[thick] (22,0.7 * 16) -- (31,0.7 * 16);        \draw[oldq,very thick] (31,0.7 * 16) -- (31,0);
        \draw[thick] (17,0.7 * 22) -- (32,0.7 * 22);        \draw[oldq,very thick] (32,0.7 * 22) -- (32,0);
        \draw[thick] (34,0.7 * 5) -- (39,0.7 * 5);        \draw[oldq,very thick] (39,0.7 * 5) -- (39,0);
        \draw[thick] (35,0.7 * 13) -- (40,0.7 * 13);        \draw[oldq,very thick] (40,0.7 * 13) -- (40,0);
        \draw[thick] (42,0.7 * 12) -- (44,0.7 * 12);        \draw[oldq,very thick] (44,0.7 * 12) -- (44,0);
        \draw[thick] (41,0.7 * 15) -- (45,0.7 * 15);        \draw[oldq,very thick] (45,0.7 * 15) -- (45,0);
        \draw[thick] (37,0.7 * 17) -- (47,0.7 * 17);        \draw[oldq,very thick] (47,0.7 * 17) -- (47,0);
        \draw[thick] (48,0.7 * 8) -- (49,0.7 * 8);
        \draw[thick] (38,0.7 * 19) -- (49,0.7 * 19);
        \draw[thick] (46,0.7 * 20) -- (49,0.7 * 20);
        \draw[thick] (36,0.7 * 21) -- (49,0.7 * 21);
        \draw[thick] (19,0.7 * 23) -- (49,0.7 * 23);
        \draw[thick] (23,0.7 * 24) -- (49,0.7 * 24);
        \draw[thick] (20,0.7 * 25) -- (49,0.7 * 25);
        \draw[thick] (43,0.7 * 26) -- (49,0.7 * 26);
        \draw[thick] (25,0.7 * 27) -- (49,0.7 * 27);
        \draw[thick] (30,0.7 * 28) -- (49,0.7 * 28);
        \draw[thick] (15,0.7 * 29) -- (49,0.7 * 29);
        \draw[thick] (33,0.7 * 30) -- (49,0.7 * 30);
        \draw[thick] (0,0.7 * 31) -- (1,0.7 * 31);
        \draw[thick] (3,0.7 * 11) -- (5,0.7 * 11);
        \draw[thick] (4,0.7 * 14) -- (6,0.7 * 14);
        \draw[thick] (2,0.7 * 18) -- (7,0.7 * 18);
        \draw[thick] (8,0.7 * 1) -- (11,0.7 * 1);
        \draw[thick] (10,0.7 * 3) -- (12,0.7 * 3);
        \draw[thick] (14,0.7 * 4) -- (16,0.7 * 4);        \draw[newq,very thick] (16,0.7 * 4) -- (16,0);
        \draw[thick] (21,0.7 * 2) -- (24,0.7 * 2);
        \draw[thick] (13,0.7 * 6) -- (27,0.7 * 6);        \draw[newq,very thick] (27,0.7 * 6) -- (27,0);
        \draw[thick] (18,0.7 * 7) -- (28,0.7 * 7);        \draw[newq,very thick] (28,0.7 * 7) -- (28,0);
        \draw[thick] (9,0.7 * 9) -- (29,0.7 * 9);        \draw[newq,very thick] (29,0.7 * 9) -- (29,0);
        \draw[thick] (26,0.7 * 10) -- (31,0.7 * 10);        \draw[newq,very thick] (31,0.7 * 10) -- (31,0);
        \draw[thick] (22,0.7 * 16) -- (32,0.7 * 16);        \draw[newq,very thick] (32,0.7 * 16) -- (32,0);
        \draw[thick] (34,0.7 * 5) -- (39,0.7 * 5);
        \draw[thick] (35,0.7 * 13) -- (40,0.7 * 13);
        \draw[thick] (42,0.7 * 12) -- (44,0.7 * 12);
        \draw[thick] (41,0.7 * 15) -- (45,0.7 * 15);
        \draw[thick] (37,0.7 * 17) -- (47,0.7 * 17);
        \draw[thick] (48,0.7 * 8) -- (49,0.7 * 8);
        \draw[thick] (38,0.7 * 19) -- (49,0.7 * 19);
        \draw[thick] (46,0.7 * 20) -- (49,0.7 * 20);
        \draw[thick] (36,0.7 * 21) -- (49,0.7 * 21);
        \draw[thick] (17,0.7 * 22) -- (32,0.7 * 22);        \draw[newq,very thick] (32,0.7 * 22) -- (49,0.7 * 22);
        \draw[thick] (19,0.7 * 23) -- (49,0.7 * 23);
        \draw[thick] (23,0.7 * 24) -- (49,0.7 * 24);
        \draw[thick] (20,0.7 * 25) -- (49,0.7 * 25);
        \draw[thick] (43,0.7 * 26) -- (49,0.7 * 26);
        \draw[thick] (25,0.7 * 27) -- (49,0.7 * 27);
        \draw[thick] (30,0.7 * 28) -- (49,0.7 * 28);
        \draw[thick] (15,0.7 * 29) -- (49,0.7 * 29);
        \draw[thick] (33,0.7 * 30) -- (49,0.7 * 30);
    \draw[->] (20,-1) -- (29,-1); \draw[->] (-1, 11 * 0.7) -- (-1,20 * 0.7);
    \draw (24.5,-2) node{время}; \draw (-2,15.5 * 0.7) node[rotate=-90]{значения};}
\caption{Ретроактивная очередь с приоритетами}
\label{fig:priorityQueue}
\end{figure}